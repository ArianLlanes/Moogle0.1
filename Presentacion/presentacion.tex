\documentclass{beamer}
\usetheme{default}
\usecolortheme{beaver}

\usepackage{xcolor}
\usepackage{amsmath, amssymb, amsfonts, amsthm, amssymb}
\usepackage{url, hyperref}
\usepackage{tikz}



% Usar plantilla en español.
\usepackage[spanish]{babel}

% Agregar citas bibliográficas
\usepackage{cite}

% Poner código fuente en latex
\usepackage{listings}
\usepackage{color}

\usepackage{listings}
\usepackage{booktabs}
\usepackage{bookmark}
\usepackage{makecell}
\usepackage{url}
\usepackage{multirow}
\usepackage{graphicx}


\usepackage{beamerthemesplit}

\definecolor{gray97}{gray}{.97}
\definecolor{gray75}{gray}{.75}
\definecolor{gray45}{gray}{.45}


% Configuración de colores personalizados
\definecolor{keywordcolor}{RGB}{0,0,128}
\definecolor{stringcolor}{RGB}{0,128,0}
\definecolor{commentcolor}{RGB}{128,128,128}
\definecolor{backgroundcolor}{RGB}{245,245,245}
\definecolor{graphcolor}{RGB}{0,128,128}

\title[Moogle!]{\LARGE Moogle!}
\author{Arián Alí Llanes Morales}
\institute[Universidad de La Habana]
{
  Matcom
}
\date{\today}

\begin{document}
\begin{frame}
  \maketitle
\end{frame}

\section{Introducción}

\begin{frame}{¿Qué es Moogle?}
  
  \textbf{Moogle!}Moogle! es un motor de búsqueda que funciona sobre una base 
de datos de archivos de textos (.txt) y que utiliza algoritmos Term Frequency (TF) 
y de Inverse Document Frequency (IDF) y otros métodos del álgebra lineal para realizar búsquedas
 de forma rápida y eficiente entre un conjunto de documentos.


\end{frame}

\begin{frame}{Acerca de su desarrollo}
  Es una aplicación web, desarrollada con {\tt .NET Core 6.0}, 
  utilizando Blazor como {\it framework} web para la interfaz gráfica, yl
 utilizando el lenguaje de programación {\tt C\#}. \\
\end{frame}


\subsection*{Sobre su uso óptimo}

\begin{frame}{Uso del script para mayor comodidad del usuario}
    \begin{center}
	En la carpeta Script el usuario encontrará un script de bash que facilitará la interacción con el proyecto
a partir de los comandos expuestos a continuación:
\\run - Para compilar y ejecutar el proyecto.
\\report - Para compilar y generar el latex del informe.
\\slides - Para compilar y generar el latex de la presentación.
\\show-report - Para comprobar que esté creado el pdf del informe, y en caso contrario, ejecutar el comando report.
\\show-slides - Para comprobar que esté creado el pdf de la presentacion, y en caso contrario, ejecutar el comando slides.
\\clean - Limpia el contenido desechable en las carpetas Presentacion e Informe.     
    \end{center}
\end{frame}



\section{Problemas frecuentes}

\begin{frame}{Consideraciones sobre el tiempo de carga}
   La primera búsqueda realizada tiende a ser más lenta que las siguientes, debido
a que es en esta primera búsqueda donde el programa carga la base de datos, proceso que toma
algunos segundos. Debido a ciertas estructuras del código las búsquedas podrían ver su tiempo de duracion aumentado
de forma directamente proporcional a la cantidad de palabras en el query (término de la búsqueda). De cualquier manera
estas situaciones deberían ser mínimas y casi imperceptibles para el usuario.


\end{document}